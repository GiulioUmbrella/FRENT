\documentclass[1_relazione.tex]{subfiles}
\begin{document}
    \section{Test e validazione}\label{sec:test-e-validazione}
    Per creare un valido ed accessibile a tutte le categorie degli utenti, sono stati usati seguenti strumenti:

    \subsection{Total Validator}
    Usato per validare le pagine HTML generate dalle pagine HTML.\newline
    Risultato: \textbf{tutte le pagine sono valide}.
    \subsection{W3C HTML Validator}
    Come fonte pi\`{u} autorevole, \`{e} stato usato il validatore del W3C.\newline
    Risultato: \textbf{tutte le pagine sono valide}.
    \subsection{W3C CSS Validator}
    Usato per verificare la validit\`{a} dei fogli di stile CSS. \newline
    Risultato: \textbf{main.css e print.css sono entrambi validi}.
    \subsection{WAVE}
    Usato per effettuare la valutazione di accessibilit\`{a} del sito con un risultato: \textbf{Nessun errore generico e nessun errore di contrasto}.
    \\Segnala un falso positivo quando nell schermata ci sono dei messaggi di errore di color rosso, mentre provando con altri strumenti, come il sito:\\ \url{https://www.colorblindness.com/coblis-color-blindness-simulator/}
    e l'estensione \textit{Colorblinding} per Chrome, il problema non susiste.
    
    \subsection{Inspect Code}
    \`{E} una funzionalit\`{a} di ispezione del codice sorgente offerta dall'IDE \textbf{PHPStorm}, che permette di elencare tutti gli errori di battitura e le eccezioni non gestite.

\end{document}
