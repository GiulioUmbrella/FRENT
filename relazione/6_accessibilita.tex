\documentclass[1_relazione.tex]{subfiles}
\begin{document}

    \section{Accessibilit\`{a}}
    Per garantire l'accessibilità del sito a quante più persone possibili, sono stati tenuti certi accorgimenti per permettere la fruizione del sito in maniera agevole.
    Sono stati tenuti i seguenti accorgimenti:
    \begin{itemize}
        \item Come supporto per gli screen reader, sono presenti dei link (nascosti all'utente vedente) per saltare i men\`{u} e andare direttamente al contenuto della pagina;

        \item Come supporto per gli screen reader, sono presenti dei link (nascosti all'utente vedente) per tornare all'inizio della pagina;

        \item I titoli della pagine sono scritti dal particolare al generale;

        \item Per le parole non in lingua italiana (ad esempio \textit{Home}) viene utilizzato l'attributo \textbf{lang} e \textbf{xml:lang} per definire la lingua della parola straniera;

        \item Le segnalazione di errore non sono basate solo sul colore del messaggio ma anche sul testo (che è il più possibile informativo);

        \item Sono presenti contrasti elevati tra scritte (titoli e paragrafi) e sfondo;

        \item La dimensione del font e l'interlinea sono aumentati rispetto a quella fornita di default dai browser;

        \item Per ogni immagine (visualizzata attraverso un elemento \textbf{img}) c'è almeno l'attributo \textbf{alt}.
        Nelle immagini di anteprima degli annunci c'è anche l'attributo \textbf{longdesc} contenente la descrizione dell'immagine, richiesta agli host per questo specifico compito;

        \item Uso di immagine tramite CSS quando non portavano contenuto aggiuntivo reale, oltre a quello decorativo (utilizzato ad esempio nella pagina per gestire errori 404).

    \end{itemize}

\end{document}