\documentclass[1_relazione.tex]{subfiles}
\begin{document}

\section{Realizzazione}
Questa sezione illustra i passi per la realizzazione di FRENT.  I criteri che abbiamo utilizzato sono:
\begin{itemize}
\item Separazione fra struttura, presentazione e comportamento 
\item Garantire l'accessibilit\`{a}
\item Scelta della corretta tecnologia da impiegare
\end{itemize}

\subsection{Base di dati}
La base di dati \`{e} stata la prima parte del progetto codificata. A partire dallo schemi ER introdotti nella sezione precedente, abbiamo creato le entit\`{a} e le relazioni in SQL. Abbiamo controllato con attenzione le politiche di reazione in caso di cancellazione e modifiche ai dati, in modo che la base di dati fosse sempre coerente. Inoltre abbiamo deciso di implementare tutte le funzionalit\`{a} di base offerte dal sito come funzioni e procedure SQL.

\subsection{Struttura}
Il punto di partenza per la creazione del sito sono state le pagine statiche. In tutte le pagine abbiamo prestato attenzione ai tag utilizzati e a rispettarne il significato semantico come visto in classe e nei laboratori. Abbiamo deciso di implementare le pagine in xhtml e utilizzare htlm5 solo per quelle pagine in cui era indispensabile. Una componente fondamentale per il funzionamento del sito sono le date il guest deve sempre indicare la data di inizio e di fine della prenotazione. Per facilitare l'inserimento di questo dato abbiamo scelto di utilizzare il widget-calendario di html5. Abbiamo giudicato improponibile chiedere al guest di inserire 6 numeri (in un formato specifico per giunta) per ogni ricerca. Le pagine di home, elenco risultati e dettagli annunci sono perci\`{o} implementate con html5. \\
Abbiamo inoltre implementato gli schemi esatti che sono stati introdotti nella sezione di progettazione realizzati sotto forma di liste. In questa fase abbiamo realizzato gli header e i footer che poi sarebbero stati scelti dinamicamente in un secondo periodo. \\
Per ragioni di accessibilit\`{a}, tutti i form sono racchiusi nel tag fieldset e utilizzano il tag legend .
Le pagine cosi composte ci hanno permesso di capire se stavamo andando nella direzione giusta. Una delle decisione che abbiamo preso \`{e} stata rimuovere una delle funzionalit\`{a} che avevamo imposta all'inizio, l'elenco degli annunci preferiti del guest (non riportato nello schema).

\subsection{Presentazione}
Il secondo passaggio \`{e} stata la presentazione. In questo paragrafo riportiamo i punti pi\`{u} rilevanti. Abbiamo creato un file .css separato per la presentazione che poi abbiamo inserito nella pagine .html. Il font utilizzato per le pagine \`{e} quello senza grazie, per migliorare la leggibilit\`{a}. Il font utilizzato per la stampa delle pagine \`{e} con grazia. Questo per garantire migliore leggibilit\`{a} dei contenuti.  \\
 % Grid
Per organizzare gli elementi nella pagina abbiamo utilizzato i grid layout. L'uso di questa tecnologia ha permesso di ottenere ottimi risultati in breve tempo. Abbiamo impostato tre tipi di grid:
\begin{itemize}
\item \textbf{Layout colonna} La pagina è formata da un'unica colonna.
\item \textbf{Layout a tre pannelli} Layout a tre pannelli.
\item \textbf{Layout centrale} La pagina è formata da un'unica colonna, ma viene data maggiore enfasi alla zona centrale. Tutti i form utilizzano questo layout
\end{itemize}

Lo svantaggio di utilizzare Grid \`{e} che non garantisce retrocompatibilt\`{a} con i browser pi\`{u} vecchi, ad esempio IE. Tuttavia come illustrato in precedenza la percentuale attesa di utenti di FRENT che usano IE \`{e} talmente bassa da giustificare l'utilizzo di Grid. In un secondo momento abbiamo introdotto alcune impostazioni css specifiche per IE. \\
% Form
Tutti i form rispettano le stesse convenzioni interne per non creare disorientamento nell'utente. Per modificare e inserire i dati abbiamo utilizzato dei form posizionati al centro della pagina. Nel momento in cui l'utente modifica i dati non ha bisogno di accedere ad altre tipologie di informazioni all'interno della pagina. Fanno eccezione i form orizzontali e verticali di ricerca annunci che per\`{o} devono essere integrati nella pagina come abbiamo spiegato nella sezione sulla progettazione.

% Liste


% Colore
Per scegliere i colori abbiamo utilizzato un pallete messa a disposizione dalla w3c. La scelta del colore \`{e} stata fatta in modo arbitrario dopo aver provato diverse combinazioni. Non abbiamo l'esperienza necessaria per poter fare una scelta precisa dei colori. Abbiamo quindi cercato di utilizzare i colori nel modo pi\`{u} semplice possibile per non creare disorientamento nell'utente.

\subsection{Comportamento}
Per gestire il comportamento abbiamo usato due tecnologie, PHP e JavaScript.

\subsubsection{PHP}
Per la rendere il sito dinamica \`{e} stato utilizzato il linguaggio PHP.
La struttura del sistema \`{e} composta da:
\begin{itemize}
    \item Class_Database.php: che permette, dato un comando SQL, di interagire con il database;
    \item Class_Frent.php: che permette di effettuare le chiamate alle procedure o funzioni presenti nel Database;
    \item Classi Entit\`{a}: sono delle classi che rappresentano le entit\`{a} del database, simile a Java Beans permettono di contenere i dati in modo compatto.
    \item Classe CredenzialiDB.php viene usato per salvare i dati di accesso al Database del server;
    \item Classe DataConstraints.php: \`{e} una classe di raccolta che viene usato per salvare i limiti di lunghezza dei dati di database, permettendo cos\`{i} la gestione centrale. In questo modo, le eventuali modifiche alla dimensione di un campo di una determinata tabella pu\`{o} essere propagata facilmente a tutto il sito.
    \item Classe ImageManager: % todo
    \item Pagine: le pagine PHP vengono create prendendo in input la pagina html statica corrispondente, e  sostituendo i vari FLAG di posizione con le informazioni reperite dal database. Queste pagine non interagiscono direttamente con il database, ma lo fanno chiamando i metodi della classe Frent.php. Anche questa scelta \`{e} dettata dalla scelta di tenere in modo centralizzato le parti che nei futuri sviluppi potrebbero subire modifiche.
    \item  Script: sono dei script php che servono per effettuare delle operazioni, una queste operazioni \`{e} il logout.
    \item classi loader: sono due load_frent.php e load_header sono delle classi contengono dei codici php ripetute in quasi tutte le pagine o script, per cui si \`{e} deciso di raccoglierli e di inserirlo in queste classi.
\end{itemize}

Le pagine PHP sono state create utilizzando meno possibile il codice HTML nei file PHP, per esempio, nella pagina miei_annunci.php la definizione dei singoli annunci \`{e} stata fatta nel file \textit{components/item_mio_annuncio.html}, in questo modo \`{e} stata garantita la pulizia del codice PHP senza inquinarla con codice HTML ad eccezione della \textit{classe load_header.php}. Nei casi in cui il codice HTML richiesto \`{e} di una sola riga, si \`{e} deciso di inserirlo direttamente.

Quando vengono lanciati degli errori dalle chiamate alle funzioni dela classe Frent.php, se l'errore \`{e} stato causato dall'inserimento errato di un valore, viene mostrato il messaggio di errore, mentre se non \`{e} stato causato dall'utente, la pagina reagisce in uno dei seguenti modi:
\begin{enumerate}
    \item l'utente viene re-indirizzato alla pagina error_page.php con eventuale messaggio di errore;
    \item l'utente viene re-indirizzato alla pagina 404.php;
    \item interviene il server e mostra l'eventuale errore.
\end{enumerate}

\subsubsection{Javascript}
Il linguaggio JavaScript è stato impiegato per effettuare i controlli lato client (direttamente dal browser) sull'input inserito dall'utente nei diversi form.
Il file \textit{form_validator.js} contiene i metodi che vengono richiamati dai form nell'attributo \textit{onsubmit}.\\
    I metodi direttamente chiamati sono i metodi che cominciano con \textit{validazione_nome_form}, ed \'{e} quest'ultimo che richiama i singoli metodi per controllare la validit\`{a} dei valori inseriti dall'utente.
    Se il valore inserito non \`{e} valido, viene inserito un elemento \textit{strong} che contiene il messaggio di errore. 


\end{document}