\documentclass[1_relazione.tex]{subfiles}
\begin{document}

\section{Realizzazione}
Questa sezione illustra i passi per la realizzazione di FRENT.  I criteri che abbiamo utilizzato sono:
\begin{itemize}
\item Separazione fra struttura, presentazione e comportamento 
\item Garantire l'accessibilità
\item Scelta della corretta tecnologia da impiegare
\end{itemize}

\subsection{Base di dati}
La base di dati è stata la prima parte del progetto codificata. A partire dallo schemi ER introdotti nella sezione precedente, abbiamo creato le entità e le relazioni in SQL. Abbiamo controllato con attenzione le politiche di reazione in caso di cancellazione e modifiche ai dati, in modo che la base di dati fosse sempre coerente. Inoltre abbiamo deciso di implementare tutte le funzionalità di base offerte dal sito come funzioni e procedure SQL. 

\subsection{Struttura}
Il punto di partenza per la creazione del sito sono state le pagine statiche. In tutte le pagine abbiamo prestato attenzione ai tag utilizzati e a rispettarne il significato semantico come visto in classe e nei laboratori. Abbiamo deciso di implementare le pagine in xhtml e utilizzare htlm5 solo per quelle pagine in cui era indispensabile. Una componente fondamentale per il funzionamento del sito sono le date il guest deve sempre indicare la data di inizio e di fine della prenotazione. Per facilitare l'inserimento di questo dato abbiamo scelto di utilizzare il widget-calendario di html5. Abbiamo giudicato improponibile chiedere al guest di inserire 6 numeri (in un formato specifico per giunta) per ogni ricerca. Le pagine di home, elenco risultati e dettagli annunci sono perciò implementate con html5. \\
Abbiamo inoltre implementato gli schemi esatti che sono stati introdotti nella sezione di progettazione realizzati sotto forma di liste. In questa fase abbiamo realizzato gli header e i footer che poi sarebbero stati scelti dinamicamente in un secondo periodo. \\
Per ragioni di accessibilità, tutti i form sono racchiusi nel tag fieldset e utilizzano il tag legend  
Le pagine cosi composte ci hanno permesso di capire se stavamo andando nella direzione giusta. Una delle decisione che abbiamo preso è stata rimuovere una delle funzionalità che avevamo imposta all'inizio, l'elenco degli annunci preferiti del guest (non riportato nello schema). 

\subsection{Presentazione}
Il secondo passaggio è stata la presentazione. In questo paragrafo riportiamo i punti più rilevanti. Abbiamo creato un file .css separato per la presentazione che poi abbiamo inserito nella pagine .html. Il font utilizzato per le pagine è quello senza grazie, per migliorare la leggibilità. Il font utilizzato per la stampa delle pagine è con grazia. Questo per garantire migliore leggibilità dei contenuti.  \\
 % Grid
Per organizzare gli elementi nella pagina abbiamo utilizzato i grid layout. L'uso di questa tecnologia ha permesso di ottenere ottimi risultati in breve tempo. Abbiamo impostato tre tipi di grid:
\begin{itemize}
\item \textbf{Layout colonna}  La pagina è formata da un'unica colonna.
\item \textbf{Layout a tre pannell} Layout a tre pannelli.
\item \textbf{Layout centrale} La pagina è formata da un'unica colonna, ma viene data maggiore enfasi alla zona centrale. Tutti i form utilizzano questo layout
\end{itemize}

Lo svantaggio di utilizzare Grid è che non garantisce retrocompatibiltà con i browser più vecchi, ad esempio IE. Tuttavia come illustrato in precedenza la percentuale attesa di utenti di FRENT che usano IE è talmente bassa da giustificare l'utilizzo di Grid. In un secondo momento abbiamo introdotto alcune impostazioni css specifiche per IE. \\
% Form
Tutti i form rispettano le stesse convenzioni interne per non creare disorientamento nell'utente. Per modificare e inserire i dati abbiamo utilizzato dei form posizionati al centro della pagina. Nel momento in cui l'utente modifica i dati non ha bisogno di accedere ad altre tipologie di informazioni all'interno della pagina. Fanno eccezione i form orizzontali e verticali di ricerca annunci che però devono essere integrati nella pagina come abbiamo spiegato nella sezione sulla progettazione. 

% Liste


% Colore
Per scegliere i colori abbiamo utilizzato un pallete messa a disposizione dalla w3c. La scelta del colore è stata fatta in modo arbitrario dopo aver provato diverse combinazioni. Non abbiamo l'esperienza necessaria per poter fare una scelta precisa dei colori. Abbiamo quindi cercato di utilizzare i colori nel modo più semplice possibile per non creare disorientamento nell'utente.

\subsection{Comportamento}
Per gestire il comportamento abbiamo usato due tecnologie, php e javascript.

\subsubsection{PHP}
Le pagine in formato html sono state utlizz


\subsubsection{Javascript}
Javascript è stato impiegato per effettuare i controlli lato client sull'input inserito dall'utente.


\end{document}